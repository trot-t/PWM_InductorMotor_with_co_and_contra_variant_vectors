%\begin{verbatim}
\documentclass[a4paper]{article}
\usepackage[14pt]{extsizes} 
\usepackage[margin=20mm]{geometry}
\geometry{
        a4paper,
        total={165mm,247mm},
        top=30mm,
        bottom=20mm 
} 
\pagenumbering{gobble}

\usepackage[T2A]{fontenc}
\usepackage[utf8]{inputenc}
\usepackage[english,russian]{babel}
\usepackage{tikz}
\usepackage[european,cuteinductors,smartlabels]{circuitikz}
\usepackage{amssymb,amsfonts,amsmath,mathtext}
\usepackage{csquotes} 
\usepackage{hyperref}


\begin{document} 

Объект управления -- трехфазный асинхронный двигатель. Питание двигателя осуществляется автономным инвертором
напряжения выполненным по мостовой схеме на IGBT-транзисторах. Управление IGBT-транзисторами производится векторной широтно-импульсной модуляцией.  

Не предполагается наличие обратной связи с двигателем.
Схема автономного инвертора и изображающего вектора напряжения $\vec{U}$ представлена на рис. \ref{ain}.

\begin{figure}[ht!]
\begin{circuitikz}[scale=1]
\ctikzset{bipoles/length=1.0cm}

\draw(1.25,2.65)node[nigbt,bodydiode](npn1){};% 1 row 
\draw (1.25,.55) node[nigbt,bodydiode](npn4){};%1row
\draw (npn1.S) -- (npn4.D);

% to get position of +-bus
\path let \p1 = (npn1.D) in node(plus)  at (0,\y1) {};
\draw (0,0) to[C] (plus);

\draw(2.75,2.65)node[nigbt,bodydiode](npn3){};% 2 row
\draw (2.75,.55) node[nigbt,bodydiode](npn6){};% 2 row
\draw (npn3.S) -- (npn6.D);

\draw (4.25,2.65)node[nigbt,bodydiode](npn5){};;% the last row 
\draw (4.25,.55) node[nigbt,bodydiode](npn2){};% the last row
\draw (npn5.S) -- (npn2.D);

\draw (plus.center) --(npn1.D) node[above]{1} -- 
	(npn3.D) node[above]{3} -- (npn5.D) node[above]{5}; % plus-bus
\draw (0,0) -- (npn4.S) node[below]{4} -- 
	(npn6.S) node[below]{6} -- (npn2.S) node[below]{2}; % minus-bus

\draw ($(npn5.S)!0.75!(npn2.D)$) node[left]{\scriptsize$C$} to[short,*-] 
++ (0.25,0) to[L,american inductor] ++ (1,0) node(C) {};    %coil C 
\draw ($(npn3.S)!0.5!(npn6.D)$) node[left]{\scriptsize$B$} to[short,*-] 
($(npn5.S)!0.5!(npn2.D)$) -- ++ (0.25,0) to[L,american inductor] ++ (1,0);  %coil B 
\draw ($(npn1.S)!0.25!(npn4.D)$) node[left]{\scriptsize$A$} to[short,*-] 
($(npn5.S)!0.25!(npn2.D)$) -- ++ (0.25,0) to[L,american inductor] ++ (1,0) node(A) {}; 

\draw (A.center)--(C.center);
\end{circuitikz}

\begin{circuitikz}[scale=1]
\newcommand{\D}{2.4}
\newcommand{\I}{1.85}
\draw[thin] (0,0) --({\D*cos(0)},{\D*sin(0)})   node(A) {} node[right] {\tiny(100)}; 
\draw[thin] (0,0) --({\D*cos(60)},{\D*sin(60)}) node(W) {} node[above right] {\tiny(110)}; 
\draw[thin] (0,0) --({\D*cos(120)},{\D*sin(120)}) node(B) {} node[above left] {\tiny(010)}; 
\draw[thin] (0,0) --({\D*cos(180)},{\D*sin(180)}) node(U) {} node[left] {\tiny(011)};
\draw[thin] (0,0) --({\D*cos(240)},{\D*sin(240)}) node(C) {} node[below left] {\tiny(001)};
\draw[thin] (0,0) --({\D*cos(300)},{\D*sin(300)}) node(V) {} node[below right] {\tiny(101)}; 
\draw[thin] (A.center) -- (W.center) -- (B.center) -- (U.center) -- (C.center) -- 
	(V.center) -- (A.center);
\draw[very thin,red,dashed] (0,0) circle ({\I});
\draw[fill, white] ({\I*cos(25)},{\I*sin(25)})  rectangle  ({\I*cos(25)+0.5},{\I*sin(25)+0.5});
\draw[->,>=latex,thick,red] (0,0) -- ({\I*cos(25)},{\I*sin(25)}) node[above right] {$\vec{U}$}; 
\end{circuitikz}
\caption{autonomous voltage inverter and space vector}
\label{ain}
\end{figure}


В симметричной трехфазной системе изображающий вектор напряжения формируется из напряжений трех фаз по формуле Парка-Горева (\cite{Gorev},\cite{Sokolovsky}): 
\begin{equation}
\vec{U} = \frac{2}{3}\left( U_A \vec{e_A} + U_B \vec{e_B} + U_C \vec{e_c} \right)
\label{base_eq}
\end{equation} 
где  $\vec{e_A}$, $\vec{e_B}$, $\vec{e_c}$ -- единичные вектора в направлении фаз.
$U_A$, $U_B$, $U_C$ -- мгновенные напряжения, измеренные в каждой фазе.  
Мгновенные значения это перпендикулярные проекции изображающего вектора на оси фаз.  


Для симметричной системы формула \ref{base_eq} может быть получена из сложения векторных равенств для изображающего вектора в трех косоугольных системах координат.  
В системе координат, образованных фазами А и В: 
$$
\vec{U} = U^A \vec{e_A} + U^B \vec{e_B}
$$ 
где $U^A, U^B$ -- контравариантные координаты (индексы вверху) есть коэффициенты линейного разложения вектора  $\vec{U}$ по векторам $\vec{e_A}$ и $\vec{e_B}$.
Перпендикулярные проекции вектора $U_A, U_B$ называются ковариантными координатами (индексы внизу).
\begin{figure}[!ht]
\centering
\begin{circuitikz}
\newcommand{\I}{3} 
\newcommand{\alfa}{40}
\newcommand{\teta}{120} 
\newcommand{\Xaxe}{4.2}
\newcommand{\Yaxe}{4.2}

\draw[very thin, ->,>=latex] (0,0) -- ({\Xaxe},0) node[below right] {$A$};
\draw[very thin, ->,>=latex] (0,0) -- ({\Xaxe*cos(\teta)},{\Xaxe*sin(\teta)}) 
	node[right] {$B$}; 
\draw[thick,red,->,>=latex] (0,0) -- ({\I*cos(\alfa)},{\I*sin(\alfa)}) 
	node[above right] {$\vec{U}$}; 
\draw[thin,dashed] ({\I*cos(\alfa)},{\I*sin(\alfa)}) -- ({\I*cos(\alfa)}, 0 ) 
	node[below] {\tiny{$U_A$}}; 
\draw[thin,dotted] ({\I*cos(\alfa)},{\I*sin(\alfa)}) -- 
	({\I*cos(\alfa) + \I*sin(\alfa)/tan(60)}, 0 ) node[below] {\tiny{$U^A$}}; 
\draw[thin,dotted] ({\I*cos(\alfa)},{\I*sin(\alfa)}) -- 
	({\I*sin(\alfa)/tan(120)}, {\I*sin(\alfa)}) node[left] {\tiny{$U^B$}}; 
\newcommand{\Oo}{(\teta-\alfa)}
\draw[thin,dashed] ({\I*cos(\alfa)},{\I*sin(\alfa)}) -- 
	({\I*cos(\Oo)*cos(\teta)},{\I*cos(\Oo)*sin(\teta)}) node[left] {\tiny{$U_B$}}; 
\end{circuitikz}\hspace{1cm} 

\begin{circuitikz}
\newcommand{\I}{2.5} 
\newcommand{\alfa}{40}
\newcommand{\teta}{120}
\newcommand{\tepa}{240} 
\newcommand{\Xaxe}{3.2}
\newcommand{\Yaxe}{3.2}

\draw[very thin, ->,>=latex] (0,0) -- ({\Xaxe*cos(\teta)},{\Xaxe*sin(\teta)}) 
	node[right] {$B$};
\draw[very thin] (0,0) -- ({\Xaxe*cos(60)},{\Xaxe*sin(60)});
\draw[very thin, ->,>=latex] (0,0) -- ({\Yaxe*cos(\tepa)},{\Yaxe*sin(\tepa)}) 
	node[right] {$C$};
\draw[very thin] (0,0) -- ({\Xaxe*cos(-60)},{\Xaxe*sin(-60)}); 
\draw[thick,red,->,>=latex] (0,0) -- ({\I*cos(\alfa)},{\I*sin(\alfa)}) 
	node[above right] {$\vec{U}$}; 
\draw[thin,dotted] ({\I*cos(\alfa)},{\I*sin(\alfa)}) -- 
	({\I*cos(\alfa) - \I*sin(\alfa)/tan(60)}, 0 ); 
\newcommand{\Oo}{(\teta-\alfa)}
\draw[thin,dashed] ({\I*cos(\alfa)},{\I*sin(\alfa)}) -- 
	({\I*cos(\Oo)*cos(\teta)},{\I*cos(\Oo)*sin(\teta)}) node[left] {\tiny{$U_B$}};
\newcommand{\Ooo}{(60-\alfa)}
\newcommand{\DD}{\I*cos(\Ooo)}
\draw[thin,dashed] ({\I*cos(\alfa)},{\I*sin(\alfa)}) -- 
	({\DD*cos(60)},{\DD*sin(60)}) node[left=-0mm] {\tiny{$U_C$}};
\newcommand{\DDD}{(\I*cos(\alfa) + \I*sin(\alfa)/tan(60))}
\draw[thin,dotted] ({\I*cos(\alfa)},{\I*sin(\alfa)}) -- 
	({\DDD*cos(60)},{\DDD*sin(60)}) node[above left=-1.5mm] {\tiny{$U^C$}};
\newcommand{\DDDD}{(\I*cos(\alfa) - \I*sin(\alfa)/tan(60))}
\draw[thin,dotted] ({\DDDD}, 0 ) -- 
	({\DDDD*cos(-60)},{\DDDD*sin(-60)}) node[below left=-1mm] {\tiny{$U_B$}}; 
\end{circuitikz} 

	\caption{Изображающий вектор в системах координат в фазах АВ и фазах BC}
\end{figure} 
В системе координат, образованных фазами B и C: $\vec{U} = U^B \vec{e_B} + U^C \vec{e_C} $ 

Eсли сложить координаты $U^B$ в этих двух системах
$$
U^B_\text{в системе фаз \tiny{AB}}\cdot \vec{e_B} + U^B_\text{в системе фаз \tiny{BC}}\cdot \vec{e_B} = 2U_B\cdot \vec{e_B}
$$
Таким образом при сложении проекций векторов в координатных системах AB, BC и CA получаем формулу (\ref{base_eq}).


Микроконтроллер управляет силовыми ключами дискретно. Состояния ключей полумостов
представлены в виде базовых векторов напряжения на рис. \ref{ain}. В обозначениях базовых векторов позиция символа обозначает фазу (ABC). 
Значение 1 соответствует тому что потенциал соответствующей фазы полумоста равен напряжению звена постоянного тока (примем это напряжение равным 1), а
значение 0 соответствует нулевому потенциалу.

Произвольный изображающий вектор напряжения есть линейная комбинация базовых векторов.  
Конец изображающего вектора лежит в центре тяжести базовых векторов, имеющих вес равный относительным продолжительностям $m_i$ нахождения системы в данном базовом векторе (рис. \ref{Isegment}). 


\begin{figure}[!ht]
\centering
\begin{tikzpicture}[scale=0.8]
\newcommand{\D}{8} % length of edge of triangle
\newcommand{\vx}{5}
\newcommand{\vy}{2}
\draw[thick] (0,0) node[below left] {(000) {\large${\bf m_1}$}} node[left] {\text{(111)~~~~~~}}
-- ({\D},0) node[below right] {{\large${\bf m_2} (100)$}} -- 
	({\D/2},{\D*sqrt(3)/2}) node[above=8] {{\large${\bf m_3} (110)$}} -- (0,0); 
   % ivector
\draw[red,very thick,->,>=latex] (0,0) -- ({\vx}, {\vy}) node(v) {};
\draw[thin] ({\vx + \vy/tan(60)},0) node[above right] {$K^\prime$} -- (\vx,\vy) -- 
	({(\vx + \vy/tan(60))/2}, {(\vx + \vy/tan(60))*sqrt(3)/2} ) 
node[right=.15cm] {K}; % countervariant at axes C and axes A
 % subscription below
\draw[very thin] (0,-0.1) -- (0,-1.5);
\draw ({\vx + \vy/tan(60)},-0.1) -- ({\vx + \vy/tan(60)},-0.8); \draw (\D,-0.1) -- (\D,-1.5);
\draw[very thin,<->,>=latex] (0,-0.4) --  ({\vx + \vy/tan(60)}, -0.4) 
	node[midway, below] {$m_2+m_3$};
\draw[very thin,<->,>=latex]  ({\vx + \vy/tan(60)}, -0.4) -- (\D, -0.4) 
	node[midway, below] {$m_1$}; 
        % the end of vector 
\draw ({\vx}, {\vy})  node[above=0.25cm] {$\vec{U}$};
\draw[thin] ({\vy/tan(60)} ,{\vy}) --  ({\vx}, {\vy}) -- ({\D - \vy/tan(60)},{\vy});
\draw[thin] ({\vx-\vy/tan(60)},0 ) -- (\vx,\vy);
\draw[thin] ({\D/2 + (\vx-\vy/tan(60))/2}, {\D*sqrt(3)/2 - (\vx-\vy/tan(60))*sqrt(3)/2}) 
	-- (\vx,\vy);

\draw[very thin]  ({\vx - \vy/tan(60)}, -1.0) -- ({\vx - \vy/tan(60)}, -1.5);
\draw[very thin,<->,>=latex]  (0,-1.2) --  ({\vx - \vy/tan(60)}, -1.2) 
	node[midway, below] {$m_2$};
\draw[very thin,<->,>=latex] ({\vx - \vy/tan(60)}, -1.3) -- (\D, -1.3) 
	node[midway, below] {$m_1+m_3$};

        %subscription at the left
\draw[very thin] ({0 - 0.1*sqrt(3)/2}, {0 + 0.1/2}) -- ({0 - 0.8*sqrt(3)/2}, {0 + 0.8/2});
\draw[very thin] ({\vy/tan(60) - 0.1*sqrt(3)/2} ,{\vy + 0.1/2}) -- 
	({\vy/tan(60) - 0.8*sqrt(3)/2} ,{\vy + 0.8/2});
\draw[very thin,<->,>=latex] ({0 - 0.75*sqrt(3)/2}, {0 + 0.75/2}) -- 
	({\vy/tan(60) - 0.75*sqrt(3)/2} ,{\vy + 0.75/2}) node[midway, above left] {$m_3$};
\draw[very thin] ({(\vx + \vy/tan(60))/2 - 0.1*sqrt(3)/2}, {(\vx + \vy/tan(60))*sqrt(3)/2 + 0.1/2})
	-- ({(\vx + \vy/tan(60))/2 - 0.8*sqrt(3)/2}, {(\vx + \vy/tan(60))*sqrt(3)/2 + 0.8/2});
\draw[very thin,<->,>=latex] ({\vy/tan(60) - 0.7*sqrt(3)/2} ,{\vy + 0.7/2}) --
         ({(\vx + \vy/tan(60))/2 - 0.7*sqrt(3)/2}, {(\vx + \vy/tan(60))*sqrt(3)/2 + 0.7/2}) 
		 node[midway, above left] {$m_2$};
\draw[very thin] ({\D/2 - 0.1*sqrt(3)/2},{\D*sqrt(3)/2 + 0.1/2}) -- 
	({\D/2 - 0.8*sqrt(3)/2},{\D*sqrt(3)/2 + 0.8/2});
\draw[very thin,<->,>=latex]  
	({(\vx + \vy/tan(60))/2 - 0.5*sqrt(3)/2}, {(\vx + \vy/tan(60))*sqrt(3)/2 + 0.5/2}) --
        ({\D/2 - 0.5*sqrt(3)/2},{\D*sqrt(3)/2 + 0.5/2}) node[midway, above left] {$m_1$};

\end{tikzpicture}
\caption{Space vector in 1st segment}
\label{Isegment}
\end{figure}


Центр тяжести лежит на прямой $KK^\prime$ и по правилу рычага Архимеда:

\noindent$m_1\times\mid\!\text{расстояние от }m_1\text{ до }KK^\prime\!\mid = (m_2+m_3)\times\mid\!\text{расстояние от }m_2m_3\text{ до }KK^\prime\!\mid$ 

\noindent --изображающий вектор есть центр тяжести \enquote{весов} базовых векторов; 

\noindent --изображающий вектор есть векторная сумма базовых векторов с учетом \enquote{весов}, т.е. контравариантных координат вектора.

Если известны перпендикулярные составляющие на оси фаз $U_A, U_B, U_C$ нет необходимости переходить к декартовым осям~$d,q$: 
$$
        \left\{
        \begin{array}{lcl}
		m_2 &=& \frac{4}{3}\left(U_A - \frac{\mid U_C\mid}{2}\right) \\
		m_3 &=& \frac{4}{3}\left(\mid U_C\mid - \;\frac{U_A}{2}\right) \\ 
                1 &=& m_1 + m_2 + m_3 
        \end{array}
        \right.
$$

%В этих формулах учтено, что напряжение звена постоянного тока равно 1.

		
-- упрощено управление электрической машиной в котором отсутствуют лишние переходы в декартову систему $d,q$ и обратно.


\begin{thebibliography}{7}
        \bibitem{Gorev}Горев А.А. Переходные процессы синхронной машины. -- М.,Л., Гос. энергетическое изд., 1950. -- 551 c.
        \bibitem{Sokolovsky}Соколовский Г.Г. Электроприводы переменного тока с частотным регулированием: Учебник для студ. высш.учеб.заведений.
                -- М. «Академия», 2007 - 272 с.
        \bibitem{Proshivka}Заливка прошивки в STM32 через USB \url{https://habr.com/post/403007/}
        \bibitem{Zagruzchik}Программа-загрузчик \url{github.com/rogerclarkmelbourne/Arduino\_STM32}
        \bibitem{MexBios}Мехбиос \url{http://www.mechatronica-pro.com/ru/catalog/software-0}
        \bibitem{controlSUITE} \url{https://www.ti.com/tool/CONTROLSUITE} 
\end{thebibliography}

\end{document}
%\end{verbatim}
